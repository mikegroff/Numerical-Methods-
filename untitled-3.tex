\documentclass[12pt]{article}
 
\usepackage[margin=1in]{geometry}
\usepackage{amsmath,amsthm,amssymb,mathtools,amsfonts}
 
\newcommand{\N}{\mathbb{N}}
\newcommand{\R}{\mathbb{R}}
\newcommand{\Z}{\mathbb{Z}}
\newcommand{\Q}{\mathbb{Q}}
\newcommand{\defeq}{\vcentcolon=}
\newcommand{\eqdef}{=\vcentcolon}
\newcommand{\overbar}[1]{\mkern 1.5mu\overline{\mkern-1.5mu#1\mkern-1.5mu}\mkern 1.5mu}

\newenvironment{theorem}[2][Theorem]{\begin{trivlist}
\item[\hskip \labelsep {\bfseries #1}\hskip \labelsep {\bfseries #2.}]}{\end{trivlist}}
\newenvironment{lemma}[2][Lemma]{\begin{trivlist}
\item[\hskip \labelsep {\bfseries #1}\hskip \labelsep {\bfseries #2.}]}{\end{trivlist}}
\newenvironment{exercise}[2][Exercise]{\begin{trivlist}
\item[\hskip \labelsep {\bfseries #1}\hskip \labelsep {\bfseries #2.}]}{\end{trivlist}}
\newenvironment{problem}[2][Problem]{\begin{trivlist}
\item[\hskip \labelsep {\bfseries #1}\hskip \labelsep {\bfseries #2.}]}{\end{trivlist}}
\newenvironment{question}[2][Question]{\begin{trivlist}
\item[\hskip \labelsep {\bfseries #1}\hskip \labelsep {\bfseries #2.}]}{\end{trivlist}}
\newenvironment{corollary}[2][Corollary]{\begin{trivlist}
\item[\hskip \labelsep {\bfseries #1}\hskip \labelsep {\bfseries #2.}]}{\end{trivlist}}
\newenvironment{answer}[2][Answer]{\begin{trivlist}
\item[\hskip \labelsep {\bfseries #1}\hskip \labelsep {\bfseries #2.}]}{\end{trivlist}}
 
\begin{document}

\title{Numerical Methods HW 1}
\author{Michael Groff\\ 
}
\date{August 25, 2017}
\maketitle

\begin{problem}{1}
\text{ }\\
 Consider a triangle with sides a, b, and c and corresponding angles $\angle ab, \angle ac, and \angle bc.$\\
 \text{ }\\
(a) Use the law of cosines, i.e., 
\[ c2 = a^2 + b^2 - 2ab \cos(\angle ab) , \]
 to calculate c if $a = 3.7, b = 5.7, and \angle ab = 79^{\circ}. $\\
 \text{ }\\
(b) Then show $c$ to its full accuracy. \\
\text{ }\\
(c) Use the law of sines, i.e., \[ \frac{\sin(\angle ab)}{c} =\frac{ \sin(\angle ac)}{b} ,\]
(d) What MATLAB command should you have used first if you wanted to save these results to the file triangle.ans? 
 
\end{problem}

\begin{problem}{2}
\text{ }\\
Calculate $ \sqrt[3]{1.2\times 10^{20} - 12^{20}i }$
 
\end{problem}

\begin{problem}{3}
\text{ }\\
 Analytically, $\cos2 \theta  = 2 \cos (\theta)^2-1$. Check whether this is also true numerically when using MATLAB by using a number of different values of $\theta$. Use MATLAB statements which make it as easy as possible to do this. 
  
\end{problem}

\begin{problem}{4}
\text{ }\\
 How would you find out information about the fix function?
 
\end{problem}

\begin{answer}{1}
\text{ }\\
(a) Using a three input function that takes the two sides and and angle, converts the angle into radians and inputting the given information into the following expression:\\
\[ \text{deg = pi/180; angle = deg*angle; c = sqrt( a*a + b*b - 2*a*b*cos(angle));} \]\\
This returns a value of 6.1751.\\
\text{ }\\
(b) By using format long we obtain 6.17508514718764.\\
\text{ }\\
(c) Using another three input function that converts the angle, calls on the previous function and then performs the following calculation:\\
\[ \begin{aligned}
\text{deg = pi/180; c = lawcos(a,b,angle); 
angle = deg*angle; angle = sin(angle);} \\
\text{
  angle = angle*b/c;
  angle = asin(angle);
  retval = angle/deg;}
  \end{aligned}
  \]
This returns a value of 64.973.\\
\text{ }\\
(d) By using diary triangle.ans first, we can save the results the that file.\\

\end{answer}

\begin{answer}{2}
\text{ }\\
Using a single input function and entering the value twelve into the following expression:
\[ \text{  retval} = (a*(10\string^ {19}) - (a\string^{20}*i))\string^(1/3); \]
This returns 1.3637e+007 - 7.6850e+006i, Note that we can reduce this to one variable by changing $1.2 \times 10^20$ to $12*10^19$.
\end{answer}

\begin{answer}{3}
\text{ }\\
By suing a single variable function that converts to radians, evaluates each expression and then takes their difference:
\[ 
\begin{aligned}
\text{
  deg = pi/180; 
  angle = deg*angle;
  costwoth = cos(2*angle);} \\
\text{
  cosqth = 2*(cos(angle))\textasciicircum2 -1;
  retval = abs(costwoth-cosqth);}
\end{aligned}
  \]
Note that we can adjust the input variable to be an array and thus calculate this comparison for a much larger number of angles.
  

\end{answer}

\begin{answer}{4}
\text{ }\\
To find more information about the fix function we can use doc fix or help fix to display information about the function, or we can see the actual MATLAB code for the function with type fix.\\ 
\end{answer}












\end{document}


